\chapter{Conclusion and Future Work}\label{sec-conclusion}
The emergence and fast development of web-based visualization technologies paves the way for a proliferation of online visual data-driven stories. In this survey, we review visual design in data-driven storytelling and also pay attention to tools that facilitate authoring visual data stories. After introducing two taxonomies respectively based on narrative genres and storytelling roles in visualization, we organize our taxonomy based on three primary components to guide visual design in composing data-driven narratives. Then we investigate narrative authoring tools to assist in crafting high quality visualization artifacts and reflect on methodologies adopted to evaluate these tools.


The research in data-driven storytelling is a growing discipline. There is a large palette of visual data storytelling techniques emerging and recurring in a multitude of forms ranging from animated infographics and videos to interactive online visualizations. However, visualization tools are not keeping up with the pace of innovation. In addition, research on understanding factors to compose high quality data stories should be extended to encompass more advanced techniques, e.g., non-linear narratives and animation. Nowadays, there are efforts being made to understand visual data-driven stories and build narrative authoring tools to facilitate storytelling. We specifically summarize a few challenges and potential research directions which are worthwhile studying further as follows.


\textbf{Understanding of stories.} Some research has demonstrated essential elements of data-driven storytelling. However, most of them analyze explicit techniques, such as text annotations to augment visualization, and timelines or progress bars to provide navigation aids. With visual data stories becoming more abundant and creative, there is certainly room for improvement and new findings. For example, little attempt has been made to understand stories from the perspective of cognitive psychology and perception. How does a compelling story engage a broad audience? Does it ignite emotions among the viewers? If yes, is there any common technique such as visualizing personal data or providing situated design? In addition, few researches have worked on interpreting non-linear narratives and adding audio narration  in data-driven storytelling, whereas many acclaimed data videos or comics \cite{inequality, Bach2018} employ these patterns and convey good stories. How to better incorporate these implicit techniques into the visual design of data-driven stories can be a potential direction for future exploration. 


\textbf{Support from authoring tools.} As mentioned in Section 4, there is an increasing trend of authoring visualization tools, whereas these tools are not keeping pace with innovation. For example, current authoring tools only provide limited animation, which makes it uncommon to see such animated transitions in practice. Besides, they fail to support guidance or warnings during the authoring process. This is mainly due to the incomplete understanding of design choices, like consistency constraints. Another important factor is the end-users who actually craft visual stories for data insight communication. They can be experts or novice users. Thus, different design considerations should be recognized in view of their expertise. For experts, how to integrate these visualization tools into their regular workflows to improve efficiency might be a major concern. While for non-experts, tool builders should provide sufficient tutorials for the use of the tool. Guidance or creation suggestions during the process are welcome as well. 

